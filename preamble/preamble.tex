% preamble.tex
\documentclass[
    a4paper,    % to be printed on A4 paper
    12pt,       % default font size
    %oneside,   % if you want your document to be onesided only
    twoside,    % if you want your document to be printed on front and back
    %bibtotoc,	% Adding Bibliography to table of contents
    onecolumn,	% write in one column
    %idxtotoc,	% index to table of contents
]{report}

% overall layout settings
% settings for overall layout:

% definition layout and print space
\setlength{\hoffset}{-1in}
\setlength{\voffset}{-0.65in}
\setlength{\oddsidemargin}{3cm} % setting the side margin to 3cm
\setlength{\evensidemargin}{\oddsidemargin} % side margin to be the same for front/back of pages
% content is horizontally centered on the page: pagewidth=21cm - 2*3cm (margins) = 15cm of text:
\setlength{\textwidth}{15cm}
\setlength{\textheight}{23cm} % 23cm of text height
% setting separation from header, footer
\setlength{\headsep}{0.75cm}
\setlength{\footskip}{30pt}
\setlength{\headheight}{28pt} % setting header height
\setlength{\parindent}{0pt} % no indentation on a new paragraph
\setlength{\parskip}{1.5ex} % vertical space between paragraphs
\setlength{\floatsep}{\baselineskip} % distance between two floats;
\setlength{\intextsep}{1.5\baselineskip} % distance between floats inserted inside the page text (using h) and the text proper.

% to set line spacing:
\usepackage{setspace} % provides commands: \doublespacing, \onehalfspacing, \singlespacing

% Define the layout of chapter titles:
\usepackage[Bjarne]{fncychap}% options are: Bjarne, Sonny, Lenny, Bjornstrup
\ChNameAsIs \ChNameVar{\raggedleft\Huge\rm} \ChRuleWidth{0.5pt}
\ChNumVar{\raggedleft\Huge}
\ChNameAsIs \ChNameVar{\raggedleft\LARGE\rm} \ChRuleWidth{0.5pt}
\ChNumVar{\raggedleft\rm\bfseries\LARGE}
\ChTitleAsIs
\ChTitleVar{\raggedleft\rm\bfseries\Huge}


% setting header/footer with fancyheaders in separate file
% fancyheaders.tex

% New commands for loading own pagestyles
\newcommand\mypagestyle[1]{\pagestyle{#1}}
\newcommand\mylisthead[1]{\fancyhead[LE,RO]{\textsl{#1}}}

% using fancyhdr to define header and footer
\usepackage{fancyhdr}
\pagestyle{fancy}
\renewcommand\sectionmark[1]{\markright{\textsl{\thesection: #1}}}
\renewcommand\chaptermark[1]{\markboth{\textsl{\chaptername\ \thechapter: #1}}{}}

% myfancy
\fancypagestyle{myfancy}{
\fancyhead{}
\fancyhead[LE]{\leftmark}
\fancyhead[RO]{\rightmark}
\fancyfoot{}
\fancyfoot[C]{}
\fancyfoot[RO,LE]{\thepage}
\renewcommand{\headrulewidth}{0.4pt}
\renewcommand{\footrulewidth}{0.4pt}}

% plain
% redefine the style 'plain', 
% 'plain' is used when calling \chapter{foo}
\fancypagestyle{plain}{
\fancyhead{}
\fancyfoot{}
\fancyfoot[LE,RO]{\thepage}
\renewcommand{\headrulewidth}{0.0pt}
\renewcommand{\footrulewidth}{0.4pt}}

% Lists, e.g. for table of content, list of figures etc.
\fancypagestyle{lists}{
\fancyhead{}
\fancyfoot{}
\fancyfoot[LE,RO]{\thepage}
\renewcommand{\headrulewidth}{0.4pt}
\renewcommand{\footrulewidth}{0.4pt}}

% References
\fancypagestyle{myreferences}{
\mypagestyle{myfancy}
\fancyhead{}
\fancyhead[LE,RO]{\textsl{\bibname}}
\renewcommand{\headrulewidth}{0.4pt}
\renewcommand{\footrulewidth}{0.4pt}}

%
% END
%
 % find fancyheader settings here

% setting listings (for code/pseudo code)
% laodlistings

% % Embed source code package
\usepackage{listings}
\lstset{language=Python}	% Identify programming language

% general command to set parameter(s)
\lstset{
     nolol=false, % prints listings in listoflistings
     % numberbychapter=true % number of listings by chapter
     extendedchars=false, % all international characters are allowed, could cause problems
     basicstyle=\small\ttfamily, % print whole listing small
     keywordstyle=\color{gray}\bfseries, % bold magenta keywords
     % identifierstyle=\bfseries, % nothing happens
     % commentstyle=\color[RGB]{0,160,0}\small\textit, % green comments
     % stringstyle=\color[RGB]{160,32,240}\ttfamily, % typewriter type for strings
     showstringspaces=false, % no special string spaces
     escapeinside={(*@}{@*)}, % for referencing a linenumber
     % fancyvrb=true % load fancy verbatims package
     % numbers=left, % numbering on left side of code
     stepnumber=1, % numbering stepsize = 1
     % numberstyle=\tiny, % tiny numbers
     % numberblanklines=false, % no numbering of blank lines
     % firstnumber=auto|last|number % first number of code
     breaklines=true % breaks lines automatically
}

\usepackage{float}
\floatstyle{ruled}
\newfloat{program}{thp}{lop}
\floatname{program}{Program}

% renew commands for listings
\renewcommand\lstlistingname{Program}			% default is Listing
\renewcommand\lstlistlistingname{List of Programs}	% default is Listings
 % find listing/code settings here

% Formats
\usepackage{textcomp}

% Numbering of headers:
\usepackage[nottoc,notlof,notlot]{tocbibind}	% Adding all titles to the toc automatically
\setcounter{secnumdepth}{4}	% x+1 section numbering
\setcounter{tocdepth}{2}	% x+1 section numbering in the table of content (toc)
%\pagenumbering{arabic}		% options are: % arabic; roman; Roman; alph; Alph
\usepackage{afterpage}
\usepackage{nextpage}

% ifthen
\usepackage{ifthen}
\usepackage{shortvrb}

% UK-English
\usepackage[english]{babel} 
\usepackage[babel]{csquotes} % correct quotation marks, e.g. \enquote{some text}; alternatively \glqq und \grqq

% Math packages
\usepackage[intlimits,sumlimits]{amsmath} % setting placements of limits
\usepackage{amssymb}
\usepackage{amstext}	
\usepackage{amsfonts}
\usepackage{mathtools} % dynamic behaviour of amsmath
\usepackage{mathrsfs} % provides rsfs fonts (i.e. for Laplace fcn)
\usepackage{amsthm}
\usepackage{units} % package for unit, e.g. $\unit[\frac{1}{8}]{m}$
\usepackage{siunitx} % e.g. \num{1e10} for scientific notation

% TikZ:
\usepackage{tikz}
\usetikzlibrary{shapes.geometric, arrows, intersections, through}
\usetikzlibrary{decorations.text, decorations.shapes, backgrounds}
% Complex graphics with tikz, e.g. flow charts:
% \usetikzlibrary{backgrounds,shapes,calc,arrows} 
\usepackage{pgfplots} % Plots with pgf
\usepackage{pgfplotstable}
\pgfplotsset{compat=1.10} % make sure we are using a version of pgfplots that can handle the features we use!
\pgfplotsset{width=6cm} % orig: 5
\pgfplotsset{height=3.6cm} % orig: 3
\pgfplotsset{grid style={solid}}

% Graphics
\usepackage{graphicx} % package for loading graphics, e.g. png, pdf, eps, etc.
% caption/subcaption for floating objects:
\usepackage{caption}
\captionsetup{%
  %format=hang,%
  indention=2em,%
  labelformat=default,%
  labelsep=colon,%
  font=onehalfspacing,%
  labelfont=bf,%
  listformat=simple,%
  position=below,%
  skip=1em,%
  hypcap=true%
}
\usepackage{subcaption}
\captionsetup[sub]{%
  indention=1em,%
  font+=small,%
  font+=onehalfspacing,%
  labelformat=parens,%
  labelsep=space,%
  subrefformat=parens,%
  skip=1ex,%
  list=false,%true,%
  hypcap=true%
}

% Colour
\usepackage{color} % package for color support
\usepackage{colortbl} % for colors in tables

% PDF packages
\usepackage[pdftex]{hyperref}
\hypersetup{pdfauthor=\name,
  pdftitle=\thesistitle,
  pdfkeywords={Thesis, \degree}}
\hypersetup{colorlinks=true,
  linkcolor=black,
  citecolor=black,
  urlcolor=black}
\usepackage[final]{pdfpages}

% URL
\usepackage{url} % hyphenation for urls

% % Bibliography hyperref
\usepackage{natbib} % required for bibliography
\bibliographystyle{plain}
%\bibliographystyle{authordate1}

% Tables
\usepackage{booktabs}
\usepackage{array}
\usepackage{multirow}
\usepackage{tabularx}
\usepackage{dcolumn}	% Decimal column alignments
\newcolumntype{d}[1]{D{.}{.}{#1}}

% Pseudocode
\usepackage[ruled,chapter]{algorithm} % plain, boxed, ruled
\usepackage{algpseudocode}
\newcommand{\Commentinline}[1]{\hfill// \textit{#1}}
\renewcommand{\algorithmiccomment}[1]{// \textit{#1}}
\algnewcommand\algorithmicoutput{\textbf{Output:}}
\algnewcommand\Output{\item[\algorithmicoutput]}

% Change orientation of single pages (useful for page filling figures/tables):
\usepackage{pdflscape} % problem when printing, so only use it for online versions of the document
\usepackage{rotating} % e.g. \begin{sidewaystable}, or \begin{turn}{angle}

% my own clearpage till odd page
\newcommand\myclearpage{\clearpage\mypagestyle{empty} {\cleartooddpage}}
% my own clearpage till odd page
\newcommand\doubleclearpage{\cleartooddpage}

% labels/cross references for 'description' lists:
\usepackage{enumitem}
\makeatletter
\def\namedlabel#1#2{\begingroup
    #2%
    \def\@currentlabel{#2}%
    \phantomsection\label{#1}\endgroup
}
\makeatother

% footnotes with label/ref, e.g.:
% some text with a new footnote\fnmark{label}
% and on its own: \fntext{label}{footnote text}
\usepackage{refcount}
\newcommand{\fnmark}[1]{\refstepcounter{footnote}\label{#1}\footnotemark[\getrefnumber{#1}]}
\newcommand{\fntext}[2]{\footnotetext[\getrefnumber{#1}]{#2}}

% for demonstration only
\usepackage{lipsum}

%
% END
%
